\section{Conclusion and Outlook}

Reflecting on the project, it was a fun project working together with Annabelle LaNoire. It must be said that it should only be seen as an art installation and not as an auditive reminder of what the plant actually needs.  So the purpose of our project 'Moody Plant' was more the interaction with and personalization of the plant in a gamified way. It was also intended to raise the awareness of the user to the fact that the plant is a slow, yet living creature comparable in some ways to pets. 
This goal was achieved and by including it into the URA Reflection Assembly Exhibition of the CMS19 master study. It reached a wide range of people and was a 'ear catcher' at the preparation of the exhibition, although Annabelle was a bit shy at the actual live streaming event. 

As an outlook, we could create two kinds of consumer products from this concept: \\
On the one hand, we could change the mapping of the sound design to have a more useful aspect for the plant itself. So that the plant not only makes sounds caused by the physical interaction with people, but that it instead makes heard of its need for water, light, a bigger pot or fresh potting soil. So the sound design concept could be used for plant sensors that communicate in a more practical way with the users.   \\
On the other hand, the optimization of the technical implementation is kept in mind. By 3D printing a plant pot with the opportunity to add a loudspeaker, the needed technical equipment and an easier way to calibrate the settings, a more consumer friendly product would be created. After the optimization, the only thing a user has to buy is the 'Sonic Plant Pot' to make their own house plant a moody plant. Additional improvements could be self-adjusting thresholds that adept to a change in surroundings to make calibration obsolete. Another possible change would be to have a power supply without needing to plug it in with a cable. It could either be solved by using batteries or maybe even with photovoltaic panels. Finally it would make sense to use only one of our devices, be it the Arduino or the Raspberry Pi.