\section{Sound Design}


The sound design for the voice of the plant is based on prosody. The plant cannot talk like a human, it communicates its needs through abstract sounds like cats do. It meows, purrs or hisses depending on what mood it is in.


\subsection{Needy Mood}
At the beginning, the plant is quiet and then after a while the needy mood is activated, because it isn't noticed or touched in a while and is kind of in need of attention. The sounds for that particular mood are whiny (going down with the voice), questioning (going up with the voice), weeping (like a baby or a cat). The intention behind that sound decision is the human consciousness of that noises and by hearing them they recognize that something is wrong, the plant is unhappy and is in need of something.  

In total, we created 13 sounds for this mood which you can hear on Github \cite{github} mentioned in the abstract. 



\subsection{Happy Mood}
The happy mood starts as soon as the plant gets attention from the people around her. It reacts on people who are close or touch the leaves of the plant. But it only stays in the happy mood as long as it is not overwhelmed with attention. While the plant sounds happy, the people hear sounds of relief, carefree and satisfied. This is mapped with the sounds of for example a happy cat purring  or as mentioned earlier a happy baby. 

For this particular mood, we created 14 short sound snippets which you can find on  Github \cite{github}.



\subsection{Angry Mood}

By giving the plant too much attention, the happy mood changes into the angry mood where the plant rejects the fondling of the people. The noises it makes in that mood are hissing, high barking and  fast and short sound phrases. This should signal the people that the  plant wants to be left alone and needs space. 
After a while with no attention, the plant starts to get needy again and switches to the first mode.

The angry mood has a variety of 12 different sounds which are chosen randomly through the patch which you can all find on  Github \cite{github}.
